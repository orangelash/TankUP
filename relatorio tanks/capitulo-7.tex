\chapter{Conclusões e Trabalho Futuro}
\label{chap:futuroC}

\section{Introdução}
\label{chap7:sec:intro}
Neste capítulo vão ser abordados trabalhos futuros na aplicação ou utilizando a aplicação como base, tais como o refinamento do conteúdo já presente, o desenvolvimento de conteúdo extra ou utilização do código para a criação de uma nova aplicação. Além disto, é feita uma conclusão acerca de todo o processo de desenvolvimento, transmitindo assim as ideias mais importantes que seria de maior importância reter acerca do desenvolvimento da \emph{TankUP}.


\section{Conclusão}
\label{chap7:sec:conc}
Em termos do desenvolvimento da aplicação, foram conseguidos todos os objetivos pretendidos, formando assim uma aplicação completa e que consegue englobar as grandes áreas do desenvolvimento de jogos.\\

Neste processo de criação foram ultrapassadas muitas dificuldades, como a aprendizagem das ferramentas, a conexão dos modelos com o motor de jogo, a conexão dos clientes com o servidor, o controlo do tabuleiro por parte de um cliente, comunicar do tabuleiro para o servidor e até a compatibilidade da aplicação com a plataforma \emph{Android}, que é suposto ser garantida pelo motor de jogo \emph{Unity}.
Para resolver estes problemas, foram consultados vários guias, mencionados na bibliografia. 

No entanto, com a realização deste projeto, foi possível aprender novos conceitos, tais como modelação, animação, design gráfico e elementos de pós-produção, e também foi possível melhorar os meus conhecimentos de ligações de rede, de programação orientada a objetos e de engenharia de \textit{software}.\\

A área do desenvolvimento de videojogos consegue englobar uma grande diversidade de conceitos fundamentais no contexto da engenharia informática, como pode ser constatado no desenvolvimento do \emph{TankUP}, formando, na minha perspectiva, um excelente tipo de projeto, uma vez que consegue conjugar um certo fator de divertimento, um interesse pessoal e uma área que se encontra em grande desenvolvimento e expansão com elementos académicos.

O desenvolvimento de jogos é uma área da informática cada vez mais disponível para o público geral, sendo possível e, na minha opinião, muitas vezes indicado que este desenvolvimento se encare como um possível meio de distribuição de conhecimento a qualquer público interessado na área de engenharia informática.

\section{Trabalho futuro e melhorias}
\label{chap7:sec:tfm}

\subsection{Refinamentos}
\label{chap7:subsec:EF}
A primeira fase de trabalhos futuros passaria por refinar a aparência e funcionamento da aplicação, de maneira a melhorar a experiência do utilizador, aumentando a qualidade da aplicação.
Esta seria a fase mais custosa e de maior duração, dado que a manutenção e melhoramento da aplicação é um processo contínuo e que tem de ser reavaliado sempre que qualquer conteúdo é adicionado e que começa logo que o processo de desenvolvimento principal da aplicação é terminado

Especificamente, na fase inicial de refinamento seria necessário resolver ou melhorar os seguintes aspetos:
\begin{enumerate}
    \item Refazer os modelos aumentando o número de polígonos, de maneira a conferir aos modelos um aspeto mais aprimorado e menos quadrado.
    \item Criar e aplicar texturas de camuflagem e de desgaste aos modelos dos veículos.
    \item Criar animações e vídeos de maior qualidade, usando tecnologias mais avançadas para a construção de cinemáticas mais realistas.
    \item Implementar um sistema de procura de partidas, de maneira a que um jogador não tenha de saber o endereço \ac{IP} de uma máquina \textit{host} para jogar.
    \item Implementar um sistema de \textit{chat} entre os dois jogadores.
    \item Implementar um tabuleiro com modelos 3D, substituindo a vista \textit{top-down} que o jogo apresenta.
    \item Melhorar o sistema de comunicação entre o cliente e o servidor, de maneira a enviar menos mensagens pela rede. 
\end{enumerate}

Com as melhorias mencionadas acima implementadas, é expectável que a qualidade da aplicação suba consideravelmente, sendo mais confortável e agradável de ser utilizada por qualquer jogador.

\subsection{Conteúdo Extra}
\label{chap7:subsec:CE}
Um dos pontos onde a indústria dos videojogos está a apostar fortemente é em conteúdo novo que é desenvolvido para um jogo já existente, adicionando novas camadas de complexidade e interesse. Este conteúdo, referido na indústria como \ac{DLC}, pode conferir um interesse e uma dimensão completamente renovada a um jogo que já tenha alguma idade, mantendo-o relevante por mais tempo. 
Este tipo de conteúdo tem várias vantagens do que simplesmente produzir um novo jogo, como por exemplo:
\begin{enumerate}
    \item O custo relacionado a produzir conteúdo para um jogo já existente é consideravelmente menor do que produzir um novo jogo de raiz.
    \item O esforço criativo de criar conteúdo para um jogo que já tem uma ideia de conteúdo fixa é muito mais baixo que criar um novo universo de jogo, com significados, contextos e conteúdos novos. Esta ideia também se aplica a sequelas.
    \item Quando um jogo é bem-sucedido, quer dizer que o seu público gostou das ideias que foram usadas para o desenvolver. Esse facto pode ser aproveitado para fazer sequelas ou conteúdo relativamente parecido com  o conteúdo original, o que é provável ter um sucesso comparável com o o que foi desenvolvido anteriormente.
\end{enumerate} 

Todo este conteúdo podia ser disponibilizado gratuitamente ao público ou requerer uma pequena taxa por parte dos jogadores, o que poderia servir para financiar o desenvolvimento de mais e melhor conteúdo.

 \subsection{Criação de uma nova aplicação}
\label{chap7:subsec:CA}
Todo o conhecimento que foi adquirido na construção desta aplicação pode ser adaptado para a produção de outros programas, com funcionalidades e objetivos radicalmente diferentes, tais como aplicações de comunicação ou com interfaces gráficas complexas. Um exemplo seria uma aplicação de comunicação entre os operários de uma fábrica e os responsáveis por abastecer os operários de materiais que usam em construções, podendo usar uma interface gráfica com modelos 3D dos materiais, de maneira a facilitar a sua identificação por parte de operadores que podem não ter treino ou capacidade de controlar uma aplicação mais complexa ou convoluta. \\ 

 Devido à modularidade que a aplicação \emph{TankUP} apresenta e, levando em conta as funcionalidades de que o motor \emph{Unity} dispõe, uma adaptação da aplicação para um propósito semelhante ao referido acima não seria uma tarefa demasiado complexa.
